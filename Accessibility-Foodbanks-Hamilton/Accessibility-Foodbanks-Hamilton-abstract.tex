\documentclass[]{elsarticle} %review=doublespace preprint=single 5p=2 column
%%% Begin My package additions %%%%%%%%%%%%%%%%%%%
\usepackage[hyphens]{url}

  \journal{Social Science and Medicine} % Sets Journal name


\usepackage{lineno} % add
\providecommand{\tightlist}{%
  \setlength{\itemsep}{0pt}\setlength{\parskip}{0pt}}

\usepackage{graphicx}
\usepackage{booktabs} % book-quality tables
%%%%%%%%%%%%%%%% end my additions to header

\usepackage[T1]{fontenc}
\usepackage{lmodern}
\usepackage{amssymb,amsmath}
\usepackage{ifxetex,ifluatex}
\usepackage{fixltx2e} % provides \textsubscript
% use upquote if available, for straight quotes in verbatim environments
\IfFileExists{upquote.sty}{\usepackage{upquote}}{}
\ifnum 0\ifxetex 1\fi\ifluatex 1\fi=0 % if pdftex
  \usepackage[utf8]{inputenc}
\else % if luatex or xelatex
  \usepackage{fontspec}
  \ifxetex
    \usepackage{xltxtra,xunicode}
  \fi
  \defaultfontfeatures{Mapping=tex-text,Scale=MatchLowercase}
  \newcommand{\euro}{€}
\fi
% use microtype if available
\IfFileExists{microtype.sty}{\usepackage{microtype}}{}
\bibliographystyle{elsarticle-harv}
\ifxetex
  \usepackage[setpagesize=false, % page size defined by xetex
              unicode=false, % unicode breaks when used with xetex
              xetex]{hyperref}
\else
  \usepackage[unicode=true]{hyperref}
\fi
\hypersetup{breaklinks=true,
            bookmarks=true,
            pdfauthor={},
            pdftitle={Changes in accessibility to food banks and food services during COVID-19 and implications for low income populations in Hamilton, Ontario},
            colorlinks=false,
            urlcolor=blue,
            linkcolor=magenta,
            pdfborder={0 0 0}}
\urlstyle{same}  % don't use monospace font for urls

\setcounter{secnumdepth}{0}
% Pandoc toggle for numbering sections (defaults to be off)
\setcounter{secnumdepth}{0}

% Pandoc citation processing

% Pandoc header
\usepackage[table]{xcolor}



\begin{document}
\begin{frontmatter}

  \title{Changes in accessibility to food banks and food services during
COVID-19 and implications for low income populations in Hamilton,
Ontario}
    \author[Some University]{Author One}
   \ead{author1@example.com} 
    \author[Another University]{Author Two\corref{1}}
   \ead{author2@example.com} 
    \author[Some University]{Author Three}
   \ead{author3@example.com} 
    \author[Some University]{Author Four}
   \ead{author4@example.com} 
      \address[Some University]{Department, Street, City, State, Zip}
    \address[Another University]{Department, Street, City, State, Zip}
      \cortext[1]{Corresponding Author}
  
  \begin{abstract}
  In this paper we analyze the changes in accessibility to food banks
  and related services before and during the COVID-19 pandemic in the
  City of Hamilton, Ontario. Food banks and services are the last line
  of support for households facing food insecurity; as such, their
  relevance cannot be ignored in the midst of the economic upheaval
  caused by the pandemic. Our analysis is based on the application of
  balanced floating catchment areas, and concentrates on households with
  lower incomes (\textless CAD40,000, approximately the Low Income
  Cutoff Value for a city of Hamilton's size). We find that
  accessibility was low to begin with in suburban and exurban parts of
  the city; furthermore, about 14\% of locations originally available in
  Hamilton closed during the pandemic, further reducing accessibility.
  The impact of closures on the level of service of the remaining
  facilities, and on accessibility, was disproportionate, with
  system-wide losses exceeding 40\%. Those losses were geographically
  and demographically uneven. While every part of the city faced a
  reduction in accessibility, inner suburbs fared worse in terms of loss
  of accessibility. As well, children (age \(\le 18\)) appear to have
  been impacted the most.
  \end{abstract}
  
 \end{frontmatter}




\end{document}

