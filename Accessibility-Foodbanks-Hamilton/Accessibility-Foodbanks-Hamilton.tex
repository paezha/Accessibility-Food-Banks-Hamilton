\documentclass[]{elsarticle} %review=doublespace preprint=single 5p=2 column
%%% Begin My package additions %%%%%%%%%%%%%%%%%%%
\usepackage[hyphens]{url}

  \journal{An awesome journal} % Sets Journal name


\usepackage{lineno} % add
\providecommand{\tightlist}{%
  \setlength{\itemsep}{0pt}\setlength{\parskip}{0pt}}

\usepackage{graphicx}
\usepackage{booktabs} % book-quality tables
%%%%%%%%%%%%%%%% end my additions to header

\usepackage[T1]{fontenc}
\usepackage{lmodern}
\usepackage{amssymb,amsmath}
\usepackage{ifxetex,ifluatex}
\usepackage{fixltx2e} % provides \textsubscript
% use upquote if available, for straight quotes in verbatim environments
\IfFileExists{upquote.sty}{\usepackage{upquote}}{}
\ifnum 0\ifxetex 1\fi\ifluatex 1\fi=0 % if pdftex
  \usepackage[utf8]{inputenc}
\else % if luatex or xelatex
  \usepackage{fontspec}
  \ifxetex
    \usepackage{xltxtra,xunicode}
  \fi
  \defaultfontfeatures{Mapping=tex-text,Scale=MatchLowercase}
  \newcommand{\euro}{€}
\fi
% use microtype if available
\IfFileExists{microtype.sty}{\usepackage{microtype}}{}
\bibliographystyle{elsarticle-harv}
\ifxetex
  \usepackage[setpagesize=false, % page size defined by xetex
              unicode=false, % unicode breaks when used with xetex
              xetex]{hyperref}
\else
  \usepackage[unicode=true]{hyperref}
\fi
\hypersetup{breaklinks=true,
            bookmarks=true,
            pdfauthor={},
            pdftitle={Changes in accessibility to food banks and food services during COVID-19 and implications for the food security of vulnerable populations in Hamilton, Ontario},
            colorlinks=false,
            urlcolor=blue,
            linkcolor=magenta,
            pdfborder={0 0 0}}
\urlstyle{same}  % don't use monospace font for urls

\setcounter{secnumdepth}{0}
% Pandoc toggle for numbering sections (defaults to be off)
\setcounter{secnumdepth}{0}

% Pandoc citation processing
\newlength{\csllabelwidth}
\setlength{\csllabelwidth}{3em}
\newlength{\cslhangindent}
\setlength{\cslhangindent}{1.5em}
% for Pandoc 2.8 to 2.10.1
\newenvironment{cslreferences}%
  {}%
  {\par}
% For Pandoc 2.11+
\newenvironment{CSLReferences}[3] % #1 hanging-ident, #2 entry spacing
 {% don't indent paragraphs
  \setlength{\parindent}{0pt}
  % turn on hanging indent if param 1 is 1
  \ifodd #1 \everypar{\setlength{\hangindent}{\cslhangindent}}\ignorespaces\fi
  % set entry spacing
  \ifnum #2 > 0
  \setlength{\parskip}{#2\baselineskip}
  \fi
 }%
 {}
\usepackage{calc} % for calculating minipage widths
\newcommand{\CSLBlock}[1]{#1\hfill\break}
\newcommand{\CSLLeftMargin}[1]{\parbox[t]{\csllabelwidth}{#1}}
\newcommand{\CSLRightInline}[1]{\parbox[t]{\linewidth - \csllabelwidth}{#1}}
\newcommand{\CSLIndent}[1]{\hspace{\cslhangindent}#1}

% Pandoc header
\usepackage[table]{xcolor}
\usepackage{booktabs}
\usepackage{longtable}
\usepackage{array}
\usepackage{multirow}
\usepackage{wrapfig}
\usepackage{float}
\usepackage{colortbl}
\usepackage{pdflscape}
\usepackage{tabu}
\usepackage{threeparttable}
\usepackage{threeparttablex}
\usepackage[normalem]{ulem}
\usepackage{makecell}
\usepackage{xcolor}



\begin{document}
\begin{frontmatter}

  \title{Changes in accessibility to food banks and food services during
COVID-19 and implications for the food security of vulnerable
populations in Hamilton, Ontario}
    \author[Some University]{Author One\corref{1}}
   \ead{author1@example.com} 
    \author[]{Author Two}
   \ead{author2@example.com} 
    \author[Some University]{Author Three\corref{2}}
   \ead{author3@example.com} 
    \author[Another University]{Author Four\corref{2}}
   \ead{author4@example.com} 
      \address[Some University]{Department, Street, City, State, Zip}
    \address[Another University]{Department, Street, City, State, Zip}
      \cortext[1]{Corresponding Author}
    \cortext[2]{Equal contribution}
  
  \begin{abstract}
  This is the abstract.

  It consists of two paragraphs.
  \end{abstract}
  
 \end{frontmatter}

\hypertarget{credit-author-statement}{%
\section{CRediT author statement}\label{credit-author-statement}}

\textbf{Author 1:} Conceptualization, Methodology, Software, Validation,
Formal analysis, Investigation, Data Curation, Writing - Original Draft,
Visualization, Supervision; \textbf{Author 2:} Conceptualization,
Methodology, Software, Validation, Formal analysis, Investigation, Data
Curation, Writing - Original Draft, Visualization; \textbf{Author 3:}
Conceptualization, Investigation, Data Curation, Writing - Original
Draft; \textbf{Author 4:} Resources - Writing: Review \& Editing -
Supervision

\newpage

\hypertarget{introduction}{%
\section{Introduction}\label{introduction}}

Food insecurity is defined as an ``inadequate or uncertain access to a
sufficient quantity and/or adequate quality of food'' due to a
household's financial limitations (Enns et al., 2020). This condition
has been associated with reductions in nutritional outcomes
(Bhattacharya et al., 2004; Kirkpatrick and Tarasuk, 2008; Olson, 1999)
and physical and mental health in children and adults (Elgar et al.,
2021; Jones, 2017; Ramsey et al., 2011; Seligman et al., 2010; Stuff et
al., 2004). Over at least the past four decades food banks and related
services have become an essential line of defense against food
insecurity in Canadian communities (Black and Seto, 2020; Holmes et al.,
2018; Riches, 2002; Tarasuk et al., 2020). In this respect, Canada is
not unlike numerous other wealthy countries where a systematic
dismantling of the welfare state took place in the intervening period
(Tarasuk et al., 2014).

The emergence of COVID-19, the worst public health crisis since the 1918
flu pandemic, has revealed important social and economic fault lines,
and pre-existing patterns of inequality appear to have been exacerbated.
Along several other dimensions of stress (e.g., accessibility to health
care facilities, Pereira et al., 2021a), this seems to be the case for
food insecurity as well (Laborde et al., 2020). In the US, for example,
it has been estimated that there was an increase of more than 30\% in
household food insecurity, and more than one third of households were
discovered to be newly food insecure - meaning they did not experience
food insecurity before the pandemic (Niles et al., 2020). In Canada, Men
and Tarasuk (2021) report that about 25\% of individuals who experienced
job insecurity (a relatively common occurrence during the pandemic),
also experienced food insecurity. Similarly, according to Statistics
Canada (2020a), in the early stages of the pandemic almost 15\% of
individuals reported living in a household that faced food insecurity;
the risk of food insecurity was substantially higher for households with
children. The difference between households with and without children
was significant, and 11.7\% of households with children indicated that
``food didn't last and {[}there was{]} no money to get more'' sometimes
or often, compared to 7.3\% of households without children; likewise,
13\% of households with children indicated that they ``{[}c{]}ouldn't
afford balanced meals'' sometimes or often, compared to 8.8\% of
households without children.

The impacts of food insecurity during the pandemic are alarming, since
diet-related diseases, such as obesity, heart-disease, and diabetes,
were already critical public health concerns in Canada prior to COVID-19
(Boucher et al., 2017). While food banks are not necessarily a stable
solution to food insecurity and in fact may encourage a retrenchment of
neoliberal policy (Wakefield et al., 2013), at least they can be argued
to provide a resource of last instance to households in precarious
situations (Bazerghi et al., 2016). As recently as 2019, the Hamilton
Hunger Report\footnote{https://www.hamiltonfoodshare.org/wp-content/uploads/Hamilton-Food-Share-Hunger-Report-2019.pdf}
noted that food banks in Hamilton, Ontario, recorded the highest number
of visitors in the past 29 years; the number of children visiting food
banks (minors up to 18 years old) was 9,125 in March 2019, up from 8,278
the year before, a rate of increase greater than population growth.

It is known that urban food environments, within which people make their
daily food choices, are essential in influencing eating behaviours and
health outcomes, based on factors such as food availability, ease of
geographic accessibility and socio-demographic variations (Paez et al.,
2010; Vanderlee and L'Abbé, 2017; Widener, 2018). However, while there
is a wealth of literature that has examined the topic of geographic
accessibility to healthy food through the ``food desert'' concept, there
has been little research into accessibility to food banks. Although
previous work has explored differences in \emph{accessing} food banks,
such as how some households utilize food banks over short periods of
time while others regularly utilize food banks as longer-term resource
(e.g., Enns et al., 2020), we are not aware of any research that has
focused on the geographic component of accessibility.

The study of place-based geographic accessibility is concerned with
capturing the potential to reach destinations of value using the
transportation network (Páez et al., 2012). Indeed, the Government of
Canada's recent Food Policy has made ``access'' to healthy food a
priority for Canadian communities\footnote{https://www.agr.gc.ca/eng/about-our-department/key-departmental-initiatives/food-policy/the-food-policy-for-canada/?id=1597863791042}
and previous survey research has suggested that such accessibility plays
a key role in user satisfaction with food bank service delivery (Holmes
et al., 2018). However, as with research into the prevalence of food
deserts, accessibility to food banks is unlikely to be evenly
distributed, and variation throughout a city can be expected due to
transportation network characteristics, and the spatial distribution of
food bank locations and the population they are meant to serve.
Furthermore, policy responses to the COVID-19 pandemic likely have added
to the distress of vulnerable households. Non-pharmaceutical
interventions during the pandemic involving restrictions in mobility
have increased the friction of travel, in particular by transit on which
low income populations are more reliant (e.g., DeWeese et al., 2020). At
the same time, the pandemic has created additional stress for the
operators of food banks through disruptions in the supply chain (e.g.,
McKay et al., 2021) as well as concerns surrounding the delivery of
service in safe conditions and possible cancellation of food service
programs.

For this study, we aim to look at how the landscape of food banks and
related services (e.g.~low-cost or free meal service providers)
available in Hamilton, Ontario, changed during the pandemic. Did the
number of open food bank services diminish? If so, what was the
accessibility to food banks before the pandemic from the perspective of
low income households, and how it changed during the pandemic? And
finally, who are most likely to have been impacted by changes in the
accessibility landscape? This paper first looks at the distribution of
food banks and related services before and during the pandemic. Then, we
use the balanced floating catchment area approach of Paez et al. (2019)
to investigate the accessibility situation. For this, we use a fully
disaggregated approach based on parcel-level data. Socio-economic and
demographic data are drawn from the latest Census of Canada (2016),
whereas travel information is from the most recent regional travel
survey from 2016. This paper follows reproducible research
recommendations (see Brunsdon and Comber, 2020), and the research was
conducted using open source tools for transportation analysis (Lovelace,
2021). The code and data necessary to reproduce the analysis are
available in a public repository\footnote{add repository}.

\hypertarget{food-insecurity-and-food-bank-use-in-canada}{%
\section{Food Insecurity and Food Bank Use in
Canada}\label{food-insecurity-and-food-bank-use-in-canada}}

Food insecurity is the inability to acquire and consume an adequate
amount or good quality food, leading to inadequate nutrient intake (Enns
et al., 2020; Kirkpatrick and Tarasuk, 2008; Tarasuk and Vogt, 2009).
This nutrient deficiency is associated with major population health
concerns, particularly among Canadians at socio-economic disadvantage
(Bazerghi et al., 2016). Data on food insecurity in Canada is collected
in several ways. Quantitatively, official government surveys such as the
Household Food Security Survey Module (HFSSM), the Canadian Community
Health Surveys (CCHS), the International Study of Adults (LISA), and
official classifications determined by Health Canada in relation to
socio-demographic variables offer some insight into food insecurity
(Gundersen et al., 2018; Kirkpatrick and Tarasuk, 2008; Tarasuk and
Vogt, 2009). Nationally, it has been found that food insecurity impacts
approximately 12.3\% of Canadian households (Tarasuk et al., 2014)

Food banks - sometimes also referred to as `food pantries' and `food
shelves' - originated as a community response to aid those with
inadequate food by voluntarily offering them meals and ingredients
(Loopstra and Tarasuk, 2012; Riches, 2002). The scope and objectives of
food banks can vary by region and by country, and these organizations
can include not only prepared meals and aliments, but also shared spaces
to connect in community gardens and community kitchens (Wakefield et
al., 2013). Although in their origin food banks were meant to be provide
a temporary solution to accommodate those in hunger due to job
retrenchments and economic downfalls since the 1980s, over time they
have evolved into a community practice to secure emergency food supplies
for those in need (Loopstra and Tarasuk, 2012; Wakefield et al., 2013).

In Canada, the number of food banks has steadily increased in the past
few decades (Wakefield et al., 2013). The largest database of food banks
and their use comes from the non-profit association Food Banks Canada
(FBC), which conducts an annual assessment through its affiliated
members. FBC's 2018 Hunger Count report\footnote{https://foodbankscanada.ca/getmedia/241fb659-05f5-44a2-9cef-56f5f51db523/HungerCount-2018\_FINAL\_EN.pdf.aspx?ext=.pdf}
(the most recent available) listed 1,830 member food banks across the
country, and found that Canadians visited food banks 1.1 million times
in March of 2018. Of those accessing food banks, certain population
characteristics tend to be over-represented compared to national totals
from the 2016 Canadian Census of Population. According to FBC's 2018
data, single-adult households represent 45\% of those utilizing food
banks despite making up 28\% of Canada's population, 19\% are
single-parent households (compared to 10\% nationally), and 35\% of
those accessing food bank services are children aged 0-18 even though
their share of Canada's national population is approximately 20\%. In
addition, 59\% of households accessing food banks list social or
disability assistance as their primary source of income. Similarly,
using data from the 2011-2012 CCHS, Tarasuk et al. (2019) found higher
odds of food insecurity amongst households relying on social assistance,
those without a university degree or with children under the age of 18,
and individuals that lived alone, renters, and those identifying as
Aboriginal. While surveys revealed that only 20 to 30 percent of those
experiencing food insecurity were found to frequent food banks in Canada
(Tarasuk et al., 2014), research from Ottawa (Enns et al., 2020) and
Vancouver (Black and Seto, 2020) suggests that long-term users tend to
be older, have health or mobility challenges, live in large households,
and are less likely to have employment income.

Economic stress caused by the COVID-19 pandemic, rising unemployment
rates, and changes in poverty levels have disrupted the food environment
and led to higher rates of food insecurity (Niles et al., 2020).
Pandemic-related food security studies in the US have found a
substantial increase in households experiencing food insecurity for the
first time, and also in households experiencing more severe food
insecurity than before (Niles et al., 2020; Wolfson and Leung, 2020).
Most recently in May 2020, Canada recorded 14.7\% of its population
living in food insecurity in the past 30 days (Statistics Canada,
2020a). Considering the negative mental and physical health effects
associated with food insecurity, increases in food insecurity rates due
to the pandemic, signal a change in the food environment with potential
damages to population health during the course of the pandemic and
beyond (Niles et al., 2020). Recent data\footnote{https://www.foodbankscanada.ca/FoodBanks/MediaLibrary/COVID-Report\_2020/A-Snapshot-of-Food-Banks-in-Canada-and-the-COVID-19-Crisis\_Exec-Sum\_EN.pdf}
from FBC showed that 52\% of member food banks reported an increase in
usage in March of 2020 when initial lockdown restrictions were put in
place across much of the country. The pandemic also created significant
staffing issues with 42\% of food banks reporting a reduction in
volunteers. However, 53\% of food banks later reported a decrease in use
into the summer of 2020 which FBC members attributed to emergency
financial support programs from the federal government. Nevertheless,
some of these benefit programs were temporary, which suggests that many
households may again turn to food bank services to meet their needs.

In terms of geography, previous research conducted at the provincial
scale using data from the 2011-2012 CCHS found that the prevalence of
food insecurity ranged across the country from 11.8\% of households in
Ontario to 41\% of households in Nunavut (Tarasuk et al., 2019).
However, although previous research has examined the characteristics of
individuals and households accessing food banks, the locational or
transportation accessibility aspect of food bank access is not well
understood. A wealth of literature examining the food desert concept
suggests that, in addition to socio-economic and demographic factors,
location and transportation networks play a key role in a household's
accessibility to healthy foods (Paez et al., 2010; Vanderlee and L'Abbé,
2017; Widener, 2018). For food banks specifically, previous qualitative
research in Ontario by Smith-Carrier et al. (2017) has noted that
``transportation can be challenging, particularly if the food bank is
situated in a remote location'' (p.~32). Particularly, it appears that
participants experience challenges with the ``inordinate amount of time
necessary to obtain food, and difficulties associated with
transportation'' (p.~39). Users of food banks, according to this
research, rely on a variety of modes of transportation to access
services. Consequently, the location of facilities matters; in the words
of an interviewee: ``I wish it {[}the food bank{]} was a little more
centrally located. Because if I didn't have a bike I'd have to walk it
all the way out there and back. I wonder about people who don't''
(p.~39).

To offer greater insight into the role of transportation and location in
food bank accessibility, this research examines how geographic
accessibility to food banks and food services changed in Hamilton during
the COVID-19 pandemic.

\hypertarget{methods-and-materials}{%
\section{Methods and Materials}\label{methods-and-materials}}

\hypertarget{methods}{%
\subsection{Methods}\label{methods}}

For the research in this paper we adopt the balanced floating catchment
area approach of Paez et al. (2019). This method for estimating
accessibility is a form of the widely-used two-stage floating catchment
area method (Luo and Wang, 2003; Radke and Mu, 2000). Floating catchment
areas are used to estimate accessibility when there are potential
congestion effects, and operate by calculating first the \emph{demand}
for spatially distributed services. The demand (usually the number of
people who require a service) is used to calculate a level of service.
In a second step, the level of service is allocated back to the
population. Demand and level of service are allocated using some form of
distance-decay to embody the geographical principle that, given a
choice, people prefer to travel less than more when reaching
destinations.

More formally, the first step of this method is as follows: \[
L_j = \frac{S_j}{\sum_{i=1}^nP_iw_{ij}}
\]

\noindent where \(S_j\) is the level of supply at location \(j\), in
simplest terms whether a service point is present (i.e., \(S_j=1\)) or
not (i.e., \(S_j=0\)); \(P_i\) is the population at location \(i\) that
demands the service; and \(w_{ij}\) is a weight, typically a function of
the distance between locations \(i\) and \(j\). \(L_j\) is the level of
service at location \(j\) and it is the inverse of the number of people
that need to be serviced.

The second step in this process is then summing the level of service
that each population unit can reach, according to the distance-decay
weight: \[
A_i = \sum_{j=1}^JL_jw_{ji}
\]

\noindent where \(A_i\) is the accessibility to the service, which is in
the same units as the level of service: as the inverse of the population
being serviced. When the population being serviced is low accessibility
is high (i.e., there is little competition for the service), and
viceversa.

Floating catchment area methods are prone to overestimation of the
population and the level of service due to multiple-counting. The
population at \(P_i\) is allocated to \emph{every} service point \(j\)
for which \(w_{ij}>0\). Similarly, the level of service at \(LOS_j\) is
allocated to \emph{every} population point for which \(w_{ji}>0\). This
inflation effect has been known for several years, and several
modifications have been proposed to mitigate it (Delamater, 2013; e.g.,
Wan et al., 2012). A definitive solution to this issue was presented by
Paez et al. (2019). In order to avoid the multiple-counting in the
summations, the population and the level of service need to be allocated
\emph{proportionally}. This is achieved by standardizing the weights as
follows: \[
w_{ij}^\text{st} = \frac{w_{ij}}{\sum_{i=1}^nw_{ij}}
\] \noindent and: \[
w_{ji}^\text{st} = \frac{w_{ji}}{\sum_{j=1}^Jw_{ji}}
\]

The standardized weights satisfy the following conditions: \[
\sum_{i=1}^nw_{ij}^\text{st}=1
\]

\noindent and: \[
\sum_{j=1}^Jw_{ji}^\text{st}=1
\]

Since the population is allocated proportionally, its value is
preserved: \[
\sum_{i=1}^nP_iw_{ij}^\text{st}=P_i
\]

\noindent as is the level of service: \[
\sum_{j=1}^JL_jw_{ji}^\text{st}=L_j
\] \#\# Study Area

\textbf{Need to provide some context here: the geography which makes
transportation challenging, i.e., due to the escarpment; the main parts
of the city: city core, inner suburbs, industrial north, suburban,
exurban; and the industrial background of the city maybe, with poverty
rates compared to other Canadian cities?}

\hypertarget{data}{%
\subsection{Data}\label{data}}

Data have been prepared for sharing as a data package\footnote{add
  repository}. The contents of the data package are described next.

\hypertarget{statistics-canada}{%
\subsubsection{Statistics Canada}\label{statistics-canada}}

Population and income statistics for 2016 were retrieved at the level of
Dissemination Areas (DAs) using the package \texttt{cancensus} (von
Bergmann et al., 2021). DAs are the smallest publicly available census
geography in Canada. Income data corresponds to the count of households
by different total income groupings.

\hypertarget{origins-residential-parcels}{%
\subsubsection{Origins: Residential
parcels}\label{origins-residential-parcels}}

We converted all recorded residential land parcels in the City of
Hamilton to points on the road network. Each point includes information
about the number of residential units in the parcel. Next, we define
low-income households as those having a total income of less than
CAD40,000, which is approximately the mid-point of the low income
cut-off (LICOs) for families in Canadian cities with populations greater
than 500,000 in 2016, to match other Census data (Statistics Canada,
2020b). We then ``populate'' each residential unit with the probability
of being a low-income household based on the counts of households by
income groups in the DA in which the parcel is located. While this
method assumes a constant probability of low-income household status for
all residential units in a DA, the parcel-level analysis affords a high
level of spatial disaggregation for the accessibility analysis.

\hypertarget{destinations-food-banks-and-food-service-locations}{%
\subsubsection{Destinations: Food Banks and Food Service
Locations}\label{destinations-food-banks-and-food-service-locations}}

The locations of food banks and related food services were obtained from
the Hamilton Public Library's Food Access Guide\footnote{http://foodaccessguide.ca/sites/default/files/partnersites/pdf/foodaccessguide.pdf}.
The guide was updated in April of 2021 to indicate any change affected
on the services due to the pandemic. This includes modified business
hours, a need to make reservations before frequenting, and locations
that have completely shut down in consequence. While some food bank
services have a specific a target population, such as prioritizing
family with young children aged between 0 and 3 or accepting only those
providing proof of low-income status through housing and utility
statements, all the food bank services indicated below are designed to
accommodate those in need of food at zero to low cost. With our focus on
food banks and food services that offer free or low-cost meals at
particular locations, we first removed services such as Meals on Wheels
and other food access services such as food box, community kitchens,
student nutrition programs, and shopping and transportation. In
addition, two free meal services held on different days at the same
location were collapsed into a single service point for the
accessibility analysis. Additional details on the operations of
individual facilities is not publicly available, and with the changes in
operations it proved unfeasible to collect it. For this reason, the
analysis to follow is of accessibility to the location of food banks and
services, but not to specific services (e.g., breakfasts vs.~food
boxes).

\begin{table}

\caption{\label{tab:table-food-bank-info}\label{tab:food-bank-info}foodbank and Food Service Information.}
\centering
\resizebox{\linewidth}{!}{
\begin{tabular}[t]{c>{\centering\arraybackslash}p{15em}cc>{\centering\arraybackslash}p{15em}}
\toprule
Type & Description & Locations Pre-COVID & Locations During COVID & Additional Notes\\
\midrule
\cellcolor{gray!6}{Congregate Dining} & \cellcolor{gray!6}{Congregate and dining programs provide low-cost meals that are enjoyed in a community setting. Transportation may be provided} & \cellcolor{gray!6}{7} & \cellcolor{gray!6}{2} & \cellcolor{gray!6}{One remaining location reduced hours during COVID}\\
Community Meals & Programs often run by volunteers that organize suppers, lunches or other get-togethers that give community residents an opportunity to meet one another in a friendly and informal atmosphere while sharing a meal & 11 & 9 & NA\\
\cellcolor{gray!6}{Food Banks} & \cellcolor{gray!6}{Food Banks and Emergency Food programs provide individuals and families with grocery items free of charge} & \cellcolor{gray!6}{27} & \cellcolor{gray!6}{26} & \cellcolor{gray!6}{One remaining location reduced hours during COVID while 4 others moved to appointment only}\\
Free Meals & Meals are provided free of charge in the community through volunteer labour and donations & 9 & 5 & One remaining location reduced hours during COVID\\
\cellcolor{gray!6}{Low-Cost Meals} & \cellcolor{gray!6}{Restaurants, cafeterias and other eating establishments operated by hospitals, senior centers or other organizations which provide reduced-cost meals for low-income people, older adults or other targeted individuals.} & \cellcolor{gray!6}{2} & \cellcolor{gray!6}{1} & \cellcolor{gray!6}{The remaining location reduced hours during COVID}\\
\bottomrule
\end{tabular}}
\end{table}

\hypertarget{routing-and-travel-time-tables}{%
\subsubsection{Routing and travel time
tables}\label{routing-and-travel-time-tables}}

Travel time tables for three modes (car, transit, walking) were computed
using the parcels as the origins and the locations of the food banks as
the destinations. For routing, the package \texttt{r5r} (Pereira et al.,
2021b) was used with a network extract for the City of Hamilton from
OpenStreetMaps and the General Transit Feed Specification from Hamilton
Street Railway, the local transit operator. For routing purposes we used
maximum values of 180 min and 10,000 m walking distance: any destination
that exceeded these thresholds was ignored. The departure time used for
routing was \textbf{TIME}.

\hypertarget{transportation-tomorrow-survey}{%
\subsubsection{Transportation Tomorrow
Survey}\label{transportation-tomorrow-survey}}

We used the Data Retrieval System of the Transportation Tomorrow Survey
(TTS)\footnote{\url{http://dmg.utoronto.ca/}} to download
cross-tabulations of: 1) primary mode of travel per trip by income by
place of residence; and 2) age by income by place of residence. These
data are from the 2016 Survey (the most recent available), and data are
geocoded at the level of Traffic Analysis Zones (TAZ) using the most
recent zoning system from 2006. Each parcel point is populated with the
proportion of trips by three modes of travel: car (as driver or
passenger), transit, and walk.

\hypertarget{expected-travel-times}{%
\subsubsection{Expected Travel Times}\label{expected-travel-times}}

Once we obtained travel time tables with population (number of
households) and proportion of trips by mode, we calculated the expected
travel time \(ett\) from each parcel \(i\) to a food bank or food
service location \(j\) as follows: \[
ett_{ij} = p^c_i\cdot tt^c_{ij} + p^t_i\cdot tt^t_{ij} + p^w_i\cdot tt^w_{ij}
\]

\noindent where \(p^k_i\) is the proportion of trips by mode \(k\) in
the TAZ of parcel \(i\), and \(tt^k_{ij}\) is the travel time from
parcel \(i\) to the food bank. In other words, the expected travel time
is the weighted sum of travel times to the food bank, with the weights
given by the expected modal split in the TAZ.

\hypertarget{results-and-discussion}{%
\section{Results and Discussion}\label{results-and-discussion}}

Figure \ref{fig:foodbanks} shows the location of food banks and services
in the City of Hamilton and their status. Before the pandemic there were
58 of which 14 (24.14\%) closed during the pandemic. As shown in the
figure, food services tend to be predominantly located in the central
parts of the city. This is not surprising: population density is high
there, and it is also the part of the city where lower income households
are more numerous in absolute and relative terms (see Figure
\ref{fig:low-income-households}). Alas, this is also the part of the
city where most of the closures during the pandemic happened.

\begin{figure}
\includegraphics[width=1\linewidth]{Accessibility-Foodbanks-Hamilton_files/figure-latex/plot-location-foodbanks-1} \caption{\label{fig:foodbanks}Location of food banks/services and operation status; the dotted box is an inset of the central part of the City of Hamilton}\label{fig:plot-location-foodbanks}
\end{figure}

\begin{figure}
\includegraphics[width=1\linewidth]{Accessibility-Foodbanks-Hamilton_files/figure-latex/plot-low-income-households-1} \caption{\label{fig:low-income-households}Number and proportion of households with incomes less than CAD40,000.}\label{fig:plot-low-income-households}
\end{figure}

To implement the accessibility calculations, we must select a
distance-decay function. In this task we find limited support in the
literature, which is mostly silent on the travel patterns of people who
visit food banks. For this reason, we opt for a simple cumulative
opportunities function as follows: \[
w_{ij}=w_{ji}=
\begin{cases}
1 & \text{ if } ett_{ij}\le \delta\\
0 & \text{ otherwise}
\end{cases}
\] \noindent where \(ett_{ij}\) is the multimodal expected travel time
as described previously, and \(\delta\) is a travel threshold. When the
expected travel time exceeds this threshold, a facility is no longer
considered accessible. Moreover, the weights are standardized for the
balanced floating catchment area approach.

Figure \ref{fig:sensitivity-analysis} shows the results of conducting
sensitivity analysis of the system-wide accessibility as we vary the
threshold (considering the situation before the pandemic). There is a
clear pattern whereby more strict values of \(\delta\) are associated
with higher levels of system-wide accessibility: while increases in
accessibility that result from decreases in the travel time window might
seem counter-intuitive, this is a result of lower \emph{congestion},
since fewer households are serviced and thus competition for the same
resources is more limited. System-wide accessibility declines with
higher values of \(\delta\): as more households are serviced, congestion
grows and the level of service declines, although this happens at a
declining rate. We are not aware of any research that explain how long
people are expected to travel for food banks, but we note that in
developing countries, access to drinking water is defined as a source
that takes less than 30 minutes to reach (round trip, see UNICEF-WHO,
2019). There is no reason why people in affluent countries should be
expected to travel more for a basic need such as food. Accordingly, we
adopt a 15-minute threshold for the analysis (representing a one-way
trip). This threshold is also approximately where the rate of change in
accessibility slows down.

\begin{figure}
\includegraphics[width=0.6\linewidth]{Accessibility-Foodbanks-Hamilton_files/figure-latex/plot-results-sensitivity-analysis-1} \caption{\label{fig:sensitivity-analysis}Accessibility as a function of threshold}\label{fig:plot-results-sensitivity-analysis}
\end{figure}

Using the 15-minute threshold, we find that the system-wide
accessibility (interpreted as a provider-to-population ratio) was 0.065
(food banks/service locations per low income household in the city)
before COVID-19, but declined to 0.037 during the pandemic. It is
striking that although almost 76\% of food facilities remained in
operation during the pandemic, there was a loss of accessibility greater
than 43\%, suggesting the location of food banks and related services
plays an important role in serving those in need.

Turning to the location of individual facilities, the levels of service
offered before and during the pandemic are shown in Figure
\ref{fig:levels-of-service}. The level of service is functionally the
inverse of the number of low-income households in the travel-mode
weighted travel time catchment area of the facilities (this is because
\(S_j=1 \forall j\), i.e., each location represents a ``capacity'' of
1). Higher values mean that a facility is expected to service fewer
households. Conversely, lower values indicate grater congestion.

The general pattern of the levels of service is similar before and
during the pandemic, with lower values in the center of the city. Three
more peripheral facilities towards the south of the city also have lower
levels of service, presumably because they are expected to service
relatively large suburban/exurban populations. During the pandemic,
however, the levels of service dropped, in some cases quite
substantially. The pattern of the losses in level of service, moreover,
is not uniform. The upper pane of Figure
\ref{fig:levels-of-service-changes} shows that the three peripheral
facilities in the southern suburban/exurban part of the city had low
levels of service to begin with, but did not see major declines during
the pandemic. Further, the inset map shows that the levels of service
deteriorated more in the central part of the city. However, the loss of
level of service was not as large in the core (where most of the food
banks/services are found), but instead was more marked in the inner ring
around the core, where facilities may have faced greater demand from
both central city and suburban populations after the closure of food
banks during the pandemic.

\begin{figure}
\includegraphics[width=1\linewidth]{Accessibility-Foodbanks-Hamilton_files/figure-latex/plot-levels-of-service-1} \caption{\label{fig:levels-of-service}Levels of service at each facility pre-COVID-19 (top panel) and during COVID-19 (bottom panel).}\label{fig:plot-levels-of-service}
\end{figure}

\begin{figure}
\includegraphics[width=1\linewidth]{Accessibility-Foodbanks-Hamilton_files/figure-latex/plot-levels-of-service-changes-1} \caption{\label{fig:levels-of-service-changes}Changes in levels of service at each facility from pre-COVID-19 to during COVID-19.}\label{fig:plot-levels-of-service-changes}
\end{figure}

To further elucidate this issue, we now turn to the results of the
accessibility analysis. As with the level of service of individual
facilities, the general pattern of accessibility before and during the
pandemic is similar. Figure \ref{fig:accessibility} reveals that,
compared with the outer rural zones, the more urban zones of the city
generally exhibit higher accessibility to food banks and food service
locations. However, the pattern is not particularly smooth - this is
largely attributable to the weighting of travel times by mode of
transportation according to the trip patterns of low-income household
respondents according to the TTS. For example, in zones where low-income
households make a high proportion of trips by walking, access to food
bank locations by walking is afforded a concomitantly high weight in our
calculations of travel time compared to transit or car travel. From
this, highly-accessible locations result from a mix of characteristics:
low-income households in locations where travel options that align with
zonal modal split are available to connect them to food bank locations
with high levels of service within 15 minutes. This seems to track with
the experience of some users of these services, as reported by
Smith-Carrier et al. (2017).

\begin{figure}

{\centering \includegraphics[width=1\linewidth]{Accessibility-Foodbanks-Hamilton_files/figure-latex/plot-accessibility-1} 

}

\caption{\label{fig:accessibility}Accessibility by traffic analysis zone pre-COVID-19 (top panel) and during COVID-19 (bottom panel).}\label{fig:plot-accessibility}
\end{figure}

We find that the accessibility landscape deteriorated substantially
during the pandemic, with accessibility dropping on average by almost
41\%, but with large variations: some zones experienced changes in
accessibility of only about 10\%, whereas the most affected zone saw a
loss of accessibility of almost 71\%. Figure
\ref{fig:accessibility-changes-with-local-i} shows the changes in
accessibility. Every zone is worse off after the closure of facilities
during the pandemic, but some parts of the city seem to have been
particularly affected. We used a local indicator of spatial
autocorrelation (Anselin, 1995) to explore the pattern of change in
accessibility. Twenty-four TAZ are flagged as having significant large
losses of accessibility (at \(p<=0.10\), without correcting for multiple
comparisons). Those zones are highlighted in the figure, where it can be
seen that they form more or less compact neighborhoods. Remarkably, the
largest (and significant) drops in accessibility are not downtown, but
located in two cases in the industrial north of the city, in one case in
an inner suburb above the escarpment (\textbf{need to describe the study
area!}), and lastly in a more suburban/exurban region in the south-west
(\textbf{share other insights about these regions, maybe based on
weighted travel patterns?}).

\begin{figure}
\includegraphics[width=1\linewidth]{Accessibility-Foodbanks-Hamilton_files/figure-latex/plot-local-i-1} \caption{\label{fig:accessibility-changes-with-local-i}Changes in accessibility from pre-COVID-19 to during COVID-19. Highlighted areas had significantly large changes in accessibility according to Local Moran's I.}\label{fig:plot-local-i}
\end{figure}

Discuss the results of aggregating the accessibility by age group (see
Table \ref{tab:accessibility-by-age}). Follow with the analysis of hours
of travel?

\begin{table}

\caption{\label{tab:table-accessibility-by-age-group}\label{tab:accessibility-by-age}Accessibility by age group among members households with incomes less than CAD40,000.}
\centering
\resizebox{\linewidth}{!}{
\begin{tabular}[t]{cccccc}
\toprule
\multicolumn{3}{c}{Population} & \multicolumn{3}{c}{Accessibility} \\
\cmidrule(l{3pt}r{3pt}){1-3} \cmidrule(l{3pt}r{3pt}){4-6}
Children (age $\le$ 18) & Adults (19-64) & Seniors (age $\ge$ 65) & Before COVID-19 & During COVID-19 & Difference\\
\midrule
\cellcolor{gray!6}{3,445 (25\%)} & \cellcolor{gray!6}{11,860 (24.8\%)} & \cellcolor{gray!6}{6,584 (25.9\%)} & \cellcolor{gray!6}{0.00229} & \cellcolor{gray!6}{0.00173} & \cellcolor{gray!6}{-0.00057}\\
4,477 (32.4\%) & 12,838 (26.9\%) & 6,847 (27\%) & 0.00707 & 0.00418 & -0.00289\\
\cellcolor{gray!6}{3,541 (25.7\%)} & \cellcolor{gray!6}{13,189 (27.6\%)} & \cellcolor{gray!6}{7,020 (27.6\%)} & \cellcolor{gray!6}{0.01580} & \cellcolor{gray!6}{0.00911} & \cellcolor{gray!6}{-0.00669}\\
2,340 (17\%) & 9,918 (20.7\%) & 4,940 (19.5\%) & 0.03985 & 0.02202 & -0.01783\\
\bottomrule
\multicolumn{6}{l}{\rule{0pt}{1em}\textit{Note: }}\\
\multicolumn{6}{l}{\rule{0pt}{1em}Population values have been rounded}\\
\end{tabular}}
\end{table}

\hypertarget{conclusions}{%
\section{Conclusions}\label{conclusions}}

\textbf{LIMITATIONS}

Words go here.

\hypertarget{references}{%
\section*{References}\label{references}}
\addcontentsline{toc}{section}{References}

\hypertarget{refs}{}
\begin{CSLReferences}{1}{0}
\leavevmode\hypertarget{ref-anselin1995local}{}%
Anselin, L., 1995. Local indicators of spatial association - LISA.
Geographical Analysis 27, 93--115.

\leavevmode\hypertarget{ref-bazerghi2016role}{}%
Bazerghi, C., McKay, F.H., Dunn, M., 2016. The role of food banks in
addressing food insecurity: A systematic review. Journal of community
health 41, 732--740.

\leavevmode\hypertarget{ref-bhattacharya2004poverty}{}%
Bhattacharya, J., Currie, J., Haider, S., 2004. Poverty, food
insecurity, and nutritional outcomes in children and adults. Journal of
health economics 23, 839--862.
doi:\url{https://doi.org/10.1016/j.jhealeco.2003.12.008}

\leavevmode\hypertarget{ref-black2020examining}{}%
Black, J.L., Seto, D., 2020. Examining patterns of food bank use over
twenty-five years in vancouver, canada. VOLUNTAS: International Journal
of Voluntary and Nonprofit Organizations 31, 853--869.
doi:\url{https://doi.org/10.1007/s11266-018-0039-2}

\leavevmode\hypertarget{ref-boucher2017ontario}{}%
Boucher, B.A., Manafò, E., Boddy, M.R., Roblin, L., Truscott, R., 2017.
The ontario food and nutrition strategy: Identifying indicators of food
access and food literacy for early monitoring of the food environment.
Health promotion and chronic disease prevention in Canada : research,
policy and practice 37, 313--319.
doi:\href{https://doi.org/10.24095/hpcdp.37.9.06}{10.24095/hpcdp.37.9.06}

\leavevmode\hypertarget{ref-brunsdon2020opening}{}%
Brunsdon, C., Comber, A., 2020. Opening practice: Supporting
reproducibility and critical spatial data science. Journal of
Geographical Systems 1--20.
doi:\href{https://doi.org/10.1007/s10109-020-00334-2}{10.1007/s10109-020-00334-2}

\leavevmode\hypertarget{ref-delamater2013spatial}{}%
Delamater, P.L., 2013. Spatial accessibility in suboptimally configured
health care systems: A modified two-step floating catchment area
(M2SFCA) metric. Health \& Place 24, 30--43.
doi:\href{https://doi.org/10.1016/j.healthplace.2013.07.012}{10.1016/j.healthplace.2013.07.012}

\leavevmode\hypertarget{ref-deweese2020tale}{}%
DeWeese, J., Hawa, L., Demyk, H., Davey, Z., Belikow, A., El-geneidy,
A., 2020. A tale of 40 cities: A preliminary analysis of equity impacts
of COVID-19 service adjustments across north america. Findings.
doi:\href{https://doi.org/10.32866/001c.13395}{10.32866/001c.13395}

\leavevmode\hypertarget{ref-elgar2021relative}{}%
Elgar, F.J., Pickett, W., Pförtner, T.-K., Gariépy, G., Gordon, D.,
Georgiades, K., Davison, C., Hammami, N., MacNeil, A.H., Da Silva, M.A.,
others, 2021. Relative food insecurity, mental health and wellbeing in
160 countries. Social Science \& Medicine 268, 113556.
doi:\url{https://doi.org/10.1016/j.socscimed.2020.113556}

\leavevmode\hypertarget{ref-enns2020experiences}{}%
Enns, A., Rizvi, A., Quinn, S., Kristjansson, E., 2020. Experiences of
food bank access and food insecurity in ottawa, canada. Journal of
Hunger \& Environmental Nutrition 15, 456--472.
doi:\url{https://doi.org/10.1080/19320248.2020.1761502}

\leavevmode\hypertarget{ref-gundersen2018food}{}%
Gundersen, C., Tarasuk, V., Cheng, J., De Oliveira, C., Kurdyak, P.,
2018. Food insecurity status and mortality among adults in ontario,
canada. PloS one 13, e0202642.

\leavevmode\hypertarget{ref-holmes2018nothing}{}%
Holmes, E., Black, J.L., Heckelman, A., Lear, S.A., Seto, D., Fowokan,
A., Wittman, H., 2018. {``Nothing is going to change three months from
now''}: A mixed methods characterization of food bank use in greater
vancouver. Social Science \& Medicine 200, 129--136.
doi:\url{https://doi.org/10.1016/j.socscimed.2018.01.029}

\leavevmode\hypertarget{ref-jones2017food}{}%
Jones, A.D., 2017. Food insecurity and mental health status: A global
analysis of 149 countries. American journal of preventive medicine 53,
264--273. doi:\url{https://doi.org/10.1016/j.amepre.2017.04.008}

\leavevmode\hypertarget{ref-kirkpatrick2008food}{}%
Kirkpatrick, S.I., Tarasuk, V., 2008. Food insecurity is associated with
nutrient inadequacies among canadian adults and adolescents. The Journal
of nutrition 138, 604--612.
doi:\url{https://doi.org/10.1093/jn/138.7.1399}

\leavevmode\hypertarget{ref-laborde2020poverty}{}%
Laborde, D., Martin, W., Vos, R., 2020. Poverty and food insecurity
could grow dramatically as COVID-19 spreads. International Food Policy
Research Institute (IFPRI), Washington, DC.

\leavevmode\hypertarget{ref-loopstra2012relationship}{}%
Loopstra, R., Tarasuk, V., 2012. The relationship between food banks and
household food insecurity among low-income toronto families. Canadian
Public Policy 38, 497--514.
doi:\href{https://doi.org/10.3138/cpp.38.4.497}{10.3138/cpp.38.4.497}

\leavevmode\hypertarget{ref-lovelace2021open}{}%
Lovelace, R., 2021. Open source tools for geographic analysis in
transport planning. Journal of Geographical Systems.
doi:\href{https://doi.org/10.1007/s10109-020-00342-2}{10.1007/s10109-020-00342-2}

\leavevmode\hypertarget{ref-luo2003measures}{}%
Luo, W., Wang, F.H., 2003. Measures of spatial accessibility to health
care in a GIS environment: Synthesis and a case study in the chicago
region. Environment and Planning B-Planning \& Design 30, 865--884.

\leavevmode\hypertarget{ref-mckay2021exploring}{}%
McKay, F.H., Bastian, A., Lindberg, R., 2021. Exploring the response of
the victorian emergency and community food sector to the COVID-19
pandemic. Journal of Hunger \& Environmental Nutrition 1--15.
doi:\href{https://doi.org/10.1080/19320248.2021.1900974}{10.1080/19320248.2021.1900974}

\leavevmode\hypertarget{ref-men2021food}{}%
Men, F., Tarasuk, V., 2021. Food insecurity amid the COVID-19 pandemic:
Food charity, government assistance and employment. Canadian Public
Policy COVID-19, e2021001.
doi:\href{https://doi.org/10.3138/cpp.2021-001}{10.3138/cpp.2021-001}

\leavevmode\hypertarget{ref-niles2020early}{}%
Niles, M.T., Bertmann, F., Belarmino, E.H., Wentworth, T., Biehl, E.,
Neff, R., 2020. The early food insecurity impacts of COVID-19. Nutrients
12, 2096.

\leavevmode\hypertarget{ref-olson1999nutrition}{}%
Olson, C.M., 1999. Nutrition and health outcomes associated with food
insecurity and hunger. The Journal of nutrition 129, 521S--524S.
doi:\url{https://doi.org/10.1093/jn/129.2.521S}

\leavevmode\hypertarget{ref-paez2019demand}{}%
Paez, A., Higgins, C.D., Vivona, S.F., 2019. Demand and level of service
inflation in floating catchment area (FCA) methods. PloS one 14,
e0218773.
doi:\href{https://doi.org/10.1371/journal.pone.0218773}{10.1371/journal.pone.0218773}

\leavevmode\hypertarget{ref-paez2010relative}{}%
Paez, A., Mercado, R.G., Farber, S., Morency, C., Roorda, M., 2010.
Relative accessibility deprivation indicators for urban settings:
Definitions and application to food deserts in montreal. Urban Studies
47, 1415--1438.
doi:\href{https://doi.org/10.1177/0042098009353626}{10.1177/0042098009353626}

\leavevmode\hypertarget{ref-paez2012measuring}{}%
Páez, A., Scott, D.M., Morency, C., 2012. Measuring accessibility:
Positive and normative implementations of various accessibility
indicators. Journal of Transport Geography 25, 141--153.
doi:\url{https://doi.org/10.1016/j.jtrangeo.2012.03.016}

\leavevmode\hypertarget{ref-pereira2021geographic}{}%
Pereira, R.H.M., Braga, C.K.V., Servo, L.M., Serra, B., Amaral, P.,
Gouveia, N., Paez, A., 2021a. Geographic access to COVID-19 healthcare
in brazil using a balanced float catchment area approach. Social Science
\& Medicine 273, 113773.
doi:\url{https://doi.org/10.1016/j.socscimed.2021.113773}

\leavevmode\hypertarget{ref-pereira2021r5r}{}%
Pereira, R.H.M., Saraiva, M., Herszenhut, D., Braga, C.K.V., Conway,
M.W., 2021b. r5r: Rapid realistic routing on multimodal transport
networks with r\textsuperscript{5} in r. Findings.
doi:\href{https://doi.org/10.32866/001c.21262}{10.32866/001c.21262}

\leavevmode\hypertarget{ref-radke2000spatial}{}%
Radke, J., Mu, L., 2000. Spatial decomposition, modeling and mapping
service regions to predict access to social programs. Annals of
Geographic Information Sciences 6, 105--112.

\leavevmode\hypertarget{ref-ramsey2011food}{}%
Ramsey, R., Giskes, K., Turrell, G., Gallegos, D., 2011. Food insecurity
among australian children: Potential determinants, health and
developmental consequences. Journal of Child Health Care 15, 401--416.
doi:\url{https://doi.org/10.1177\%2F1367493511423854}

\leavevmode\hypertarget{ref-riches2002food}{}%
Riches, G., 2002. Food banks and food security: Welfare reform, human
rights and social policy. Lessons from canada? Social Policy \&
Administration 36, 648--663.

\leavevmode\hypertarget{ref-seligman2010food}{}%
Seligman, H.K., Laraia, B.A., Kushel, M.B., 2010. Food insecurity is
associated with chronic disease among low-income NHANES participants.
The Journal of nutrition 140, 304--310.

\leavevmode\hypertarget{ref-smith2017food}{}%
Smith-Carrier, T., Ross, K., Kirkham, J., Decker Pierce, B., 2017.
{`Food is a right\ldots{} nobody should be starving on our streets'}:
Perceptions of food bank usage in a mid-sized city in ontario, canada.
Journal of Human Rights Practice 9, 29--49.

\leavevmode\hypertarget{ref-statisticscanada2020food}{}%
Statistics Canada, 2020a. Food insecurity during the COVID-19 pandemic
(No. Catalogue no. 45280001).

\leavevmode\hypertarget{ref-statisticscanada2020licos}{}%
Statistics Canada, 2020b. Table 11-10-0241-01 low income cut-offs
(LICOs) before and after tax by community size and family size, in
current dollars. doi:\url{https://doi.org/10.25318/1110024101-eng}

\leavevmode\hypertarget{ref-stuff2004household}{}%
Stuff, J.E., Casey, P.H., Szeto, K.L., Gossett, J.M., Robbins, J.M.,
Simpson, P.M., Connell, C., Bogle, M.L., 2004. Household food insecurity
is associated with adult health status. The Journal of nutrition 134,
2330--2335.
doi:\href{https://doi.org/Oxford\%20University\%20Press}{Oxford University Press}

\leavevmode\hypertarget{ref-tarasuk2014food}{}%
Tarasuk, V., Dachner, N., Loopstra, R., 2014. Food banks, welfare, and
food insecurity in canada. British Food Journal.

\leavevmode\hypertarget{ref-tarasuk2020relationship}{}%
Tarasuk, V., Fafard St-Germain, A.-A., Loopstra, R., 2020. The
relationship between food banks and food insecurity: Insights from
canada. VOLUNTAS: International Journal of Voluntary and Nonprofit
Organizations 31, 841--852.
doi:\href{https://doi.org/10.1007/s11266-019-00092-w}{10.1007/s11266-019-00092-w}

\leavevmode\hypertarget{ref-tarasuk2019geographic}{}%
Tarasuk, V., Fafard St-Germain, A.-A., Mitchell, A., 2019. Geographic
and socio-demographic predictors of household food insecurity in canada,
2011--12. BMC Public Health 19, 12.
doi:\href{https://doi.org/10.1186/s12889-018-6344-2}{10.1186/s12889-018-6344-2}

\leavevmode\hypertarget{ref-tarasuk2009household}{}%
Tarasuk, V., Vogt, J., 2009. Household food insecurity in ontario.
Canadian Journal of Public Health 100, 184--188.
doi:\href{https://doi.org/10.1007/BF03405537}{10.1007/BF03405537}

\leavevmode\hypertarget{ref-world2019progress}{}%
UNICEF-WHO, 2019. Progress on household drinking water, sanitation and
hygiene 2000--2017. Special focus on inequalities.

\leavevmode\hypertarget{ref-vanderlee2017food}{}%
Vanderlee, L., L'Abbé, M., 2017. Food for thought on food environments
in canada. Health promotion and chronic disease prevention in Canada :
research, policy and practice 37, 263--265.
doi:\href{https://doi.org/10.24095/hpcdp.37.9.01}{10.24095/hpcdp.37.9.01}

\leavevmode\hypertarget{ref-vonBergmann2021cancensus}{}%
von Bergmann, J., Shkolnik, D., Jacobs, A., 2021. Cancensus: R package
to access, retrieve, and work with canadian census data and geography.

\leavevmode\hypertarget{ref-wakefield2013sweet}{}%
Wakefield, S., Fleming, J., Klassen, C., Skinner, A., 2013. Sweet
charity, revisited: Organizational responses to food insecurity in
hamilton and toronto, canada. Critical Social Policy 33, 427--450.

\leavevmode\hypertarget{ref-wan2012three}{}%
Wan, N., Zou, B., Sternberg, T., 2012. A three-step floating catchment
area method for analyzing spatial access to health services.
International Journal of Geographical Information Science 26,
1073--1089.
doi:\href{https://doi.org/10.1080/13658816.2011.624987}{10.1080/13658816.2011.624987}

\leavevmode\hypertarget{ref-widener2018spatial}{}%
Widener, M.J., 2018. Spatial access to food: Retiring the food desert
metaphor. Physiology \& behavior 193, 257--260.
doi:\url{https://doi.org/10.1016/j.physbeh.2018.02.032}

\leavevmode\hypertarget{ref-wolfson2020food}{}%
Wolfson, J.A., Leung, C.W., 2020. Food insecurity and COVID-19:
Disparities in early effects for US adults. Nutrients 12, 1648.

\end{CSLReferences}


\end{document}

