\documentclass[]{elsarticle} %review=doublespace preprint=single 5p=2 column
%%% Begin My package additions %%%%%%%%%%%%%%%%%%%
\usepackage[hyphens]{url}

  \journal{An awesome journal} % Sets Journal name


\usepackage{lineno} % add
\providecommand{\tightlist}{%
  \setlength{\itemsep}{0pt}\setlength{\parskip}{0pt}}

\usepackage{graphicx}
\usepackage{booktabs} % book-quality tables
%%%%%%%%%%%%%%%% end my additions to header

\usepackage[T1]{fontenc}
\usepackage{lmodern}
\usepackage{amssymb,amsmath}
\usepackage{ifxetex,ifluatex}
\usepackage{fixltx2e} % provides \textsubscript
% use upquote if available, for straight quotes in verbatim environments
\IfFileExists{upquote.sty}{\usepackage{upquote}}{}
\ifnum 0\ifxetex 1\fi\ifluatex 1\fi=0 % if pdftex
  \usepackage[utf8]{inputenc}
\else % if luatex or xelatex
  \usepackage{fontspec}
  \ifxetex
    \usepackage{xltxtra,xunicode}
  \fi
  \defaultfontfeatures{Mapping=tex-text,Scale=MatchLowercase}
  \newcommand{\euro}{€}
\fi
% use microtype if available
\IfFileExists{microtype.sty}{\usepackage{microtype}}{}
\bibliographystyle{elsarticle-harv}
\ifxetex
  \usepackage[setpagesize=false, % page size defined by xetex
              unicode=false, % unicode breaks when used with xetex
              xetex]{hyperref}
\else
  \usepackage[unicode=true]{hyperref}
\fi
\hypersetup{breaklinks=true,
            bookmarks=true,
            pdfauthor={},
            pdftitle={Changes in accessibility to foodbanks during COVID-19 and implications for the food security of vulnerable populations in Hamilton, Ontario},
            colorlinks=false,
            urlcolor=blue,
            linkcolor=magenta,
            pdfborder={0 0 0}}
\urlstyle{same}  % don't use monospace font for urls

\setcounter{secnumdepth}{0}
% Pandoc toggle for numbering sections (defaults to be off)
\setcounter{secnumdepth}{0}

% Pandoc citation processing
\newlength{\csllabelwidth}
\setlength{\csllabelwidth}{3em}
\newlength{\cslhangindent}
\setlength{\cslhangindent}{1.5em}
% for Pandoc 2.8 to 2.10.1
\newenvironment{cslreferences}%
  {}%
  {\par}
% For Pandoc 2.11+
\newenvironment{CSLReferences}[3] % #1 hanging-ident, #2 entry spacing
 {% don't indent paragraphs
  \setlength{\parindent}{0pt}
  % turn on hanging indent if param 1 is 1
  \ifodd #1 \everypar{\setlength{\hangindent}{\cslhangindent}}\ignorespaces\fi
  % set entry spacing
  \ifnum #2 > 0
  \setlength{\parskip}{#2\baselineskip}
  \fi
 }%
 {}
\usepackage{calc} % for calculating minipage widths
\newcommand{\CSLBlock}[1]{#1\hfill\break}
\newcommand{\CSLLeftMargin}[1]{\parbox[t]{\csllabelwidth}{#1}}
\newcommand{\CSLRightInline}[1]{\parbox[t]{\linewidth - \csllabelwidth}{#1}}
\newcommand{\CSLIndent}[1]{\hspace{\cslhangindent}#1}

% Pandoc header



\begin{document}
\begin{frontmatter}

  \title{Changes in accessibility to foodbanks during COVID-19 and
implications for the food security of vulnerable populations in
Hamilton, Ontario}
    \author[Some University]{Author One\corref{1}}
   \ead{author1@example.com} 
    \author[]{Author Two}
   \ead{author2@example.com} 
    \author[Some University]{Author Three\corref{2}}
   \ead{author3@example.com} 
    \author[Another University]{Author Four\corref{2}}
   \ead{author4@example.com} 
      \address[Some University]{Department, Street, City, State, Zip}
    \address[Another University]{Department, Street, City, State, Zip}
      \cortext[1]{Corresponding Author}
    \cortext[2]{Equal contribution}
  
  \begin{abstract}
  This is the abstract.

  It consists of two paragraphs.
  \end{abstract}
  
 \end{frontmatter}

\newpage

\hypertarget{introduction}{%
\section{Introduction}\label{introduction}}

Over at least the past four decades food banks have become an essential
line of defense against food insecurity in Canadian communities (Black
and Seto, 2020; Riches, 2002; Tarasuk et al., 2020). In this respect,
Canada is not unlike numerous other wealthy countries where a systematic
dismantling of the welfare state took place in the intervening period
(Tarasuk et al., 2014).

The emergence of COVID-19, the worst public health crisis since the 1918
flu pandemic, has revealed important social and economic fault lines,
and pre-existing patterns of inequality appear to have been exacerbated.
Along several other dimensions of stress {[}e.g., accessibility to
health care facilities; Pereira et al. (2021){]}, this seems to be the
case for food insecurity as well (Laborde et al., 2020). In the US, for
example, it has been estimated that there was an increase of more than
30\% in household food insecurity, and more than one third of households
were discovered to be newly food insecure - meaning they did not
experience food insecurity before the pandemic (Niles et al., 2020). In
Canada, Men and Tarasuk (2021) report that about 25\% of individuals who
experienced job insecurity (a relatively common occurrence during the
pandemic), also experienced food insecurity. Similarly, according to
Statistics Canada (2020), in the early stages of the pandemic almost
15\% of individuals reported living in a household that faced food
insecurity; the risk of food insecurity was substantially higher for
households with children. The difference between households with and
without children was significant, and 11.7\% of households with children
indicated that ``food didn't last and {[}there was{]} no money to get
more'' sometimes or often, compared to 7.3\% of households without
children); likewise, 13\% of households with children indicated that
they ``{[}c{]}ouldn't afford balanced meals'' sometimes or often,
compared to 8.8\% of households without children.

The impacts of food insecurity during the pandemic are alarming, since
diet-related diseases, such as obesity, heart-disease, and diabetes,
were already critical public health concerns in Canada prior to COVID-19
(Boucher et al., 2017). While foodbanks are not necessarily a stable
solution to food insecurity and in fact may encourage a retrenchment of
neoliberal policy (Wakefield et al., 2013), at least can be argued to
provide a resource of last instance to households in precarious
situations (Bazerghi et al., 2016). As recently as 2019, the Hamilton
Hunger Report\footnote{https://www.hamiltonfoodshare.org/wp-content/uploads/Hamilton-Food-Share-Hunger-Report-2019.pdf}
noted that food banks in Hamilton, Ontario, recorded the highest number
of visitors in the past 29 years; the number of children visiting
foodbanks (minors up to 18 years old) was 9,125 in March 2019, up from
8,278 the year before. It is known that the urban food environment,
within which people make their daily food choices, is essential in
influencing eating behaviours and health outcomes, based on factors such
as food availability, ease of accessibility and socio-demographic
variations (Paez et al., 2010; Vanderlee and L'Abbé, 2017). To add to
the distress of vulnerable households, non-pharmaceutical interventions
during the pandemic involving restrictions in mobility increased the
friction of travel, in particular by transit on which they are more
likely to be reliant (e.g., DeWeese et al., 2020); while at the same
time creating stress for the operators of foodbanks through disruptions
in the supply chain (e.g., McKay et al., 2021), in addition to concerns
around delivery of service in safe conditions.

For this study, we aim to look at how the landscape of food bank
services available in Hamilton, Ontario, has changed before and during
the pandemic. Have the number of open food bank services diminished? If
so, what was the accessibility to foodbanks before and during the
pandemic, from the perspective of low income households? And finally,
who are most likely to have been impacted by changes to the
accessibility landscape? This paper will first look at the distribution
of foodbanks and related services before and during the pandemic. Then,
we use the balanced floating catchment areas approach of Paez et al.
(Paez et al., 2019) to investigate the accessibility situation. We use a
fully disaggregated approach based on parcel-level data. Socio-economic
and demographic data are drawn from the latest Census of Canada (2016),
whereas travel information is from the most recent regional travel
survey from 2016. This paper follows reproducible research
recommendations (see Brunsdon and Comber, 2020), and the research was
conducted using open source tools for transportation analysis (Lovelace,
2021). The code and data necessary to reproduce the analysis are
available in a public repository\footnote{add repository}.

\hypertarget{literature-review}{%
\section{Literature Review}\label{literature-review}}

\hypertarget{food-insecurity}{%
\subsection{Food Insecurity}\label{food-insecurity}}

Food insecurity is the inability to acquire and consume an adequate
amount or good quality food, leading to inadequate nutrient intake
(Kirkpatrick and Tarasuk 2008; Tarasuk and Vogt 2009). This nutrient
deficiency has been causing major health concerns in Canadians, and
particularly those who are at a socioeconomic disadvantage (Bazerghi,
McKay, and Dunn 2016). Previous studies have aimed to look at the
relationship between the built food environment and sociodemographic
characteristics with qualitative and quantitative, or a mixed -method
approach. Quantitatively, official governmental surveys have been able
to assist with data, such as the Household Food Security Survey Module
(HFSSM), the Canadian Community Health Surveys (CCHS) , the Longitudinal
and International Study of Adults (LISA), and official classifications
determined by Health Canada in relation to sociodemographic variables
(El-hajj and Benhin 2021; Gundersen et al.~2018; Kirkpatrick and Tarasuk
2008; Tarasuk and Vogt 2009). Studies have also aimed to assess food
availability of healthy foods (e.g., fruit and vegetables) at
supermarkets in relation to sociodemographic characteristics and
geographic accessibility (Latham and Moffat 2007). In terms of findings,
studies have generated inconsistent relationships between their
evaluated availability of food and sociodemographic characteristics that
go in hand.

\hypertarget{food-banks}{%
\subsection{Food Banks}\label{food-banks}}

The number of food banks has been increasing steadily in Canada
(Wakefield et al.~2013). What was supposed to be a temporary solution to
accommodate those in hunger due to job retrenchments and economic
downfalls since the 1980s, has now developed into a community practice
to secure emergency food supplies for those in need (Loopstra and
Tarasuk 2012; Wakefield et al.~2013). Food banks were developed as a
part of a community response to aide those with inadequate food by
voluntarily giving meals and ingredients away (Loopstra and Tarasuk
2012; Riches 2002). The scope and objectives of food banks can vary by
region and by country. These organizations can include not only prepared
meals and aliments, but also shared spaces to connect in community
gardens and community kitchens (Wakefield et al.~2013) or referred to as
`food pantry' and `food shelf,' where many frequent as their primary
location to get food (Bazerghi et al.~2016). However, surveys revealed
that only 20 to 30 percent of those experiencing food insecurity were
found to frequent food banks in Canada (Tarasuk et al.~2014).

Previous studies also question if food banks are able to offer a rounded
nutritious supply of food, and if food banks are a sustainable practice
for those in need to continuously obtain their food from (Bazerghi et
al.~2016; Riches 2002).

\hypertarget{food-insecurity-in-canada-during-the-covid-19}{%
\subsection{Food Insecurity in Canada during the
COVID-19}\label{food-insecurity-in-canada-during-the-covid-19}}

Currently with the COVID-19 pandemic, disrupted economies, rising
unemployment rates, and alarming poverty levels have disrupted the food
environment by causing higher rates of food insecurity (Niles et
al.~2020). In 2012, 12.4\% of Canadians households and 11.8\% Ontarian
households experienced some degree of food insecurity (Gundersen et
al.~2018; Tarasuk, Fafard St-Germain, and Mitchell 2019). Recent
COVID-19 related food security studies in the US have found a massive
increase in households experiencing food insecurity for the first time,
and also in households experiencing more severe food insecurity than
before (Niles et al.~2020; Wolfson and Leung 2020). Most recently in May
2020, Canada recorded 14.7\% of its population living in food insecurity
in the past 30 days(Statistics Canada 2020). Food insecurity is a highly
concerning public health issue due the vast health consequences that it
can induce. Recent publishing suggests that food insecurity in adults
could lead to experiencing more stressful events (El-hajj and Benhin
2021). These increases in food insecurity rates due to the pandemic,
signal a change in the food environment with potential damages to health
outcomes in populations (Niles et al.~2020)

\hypertarget{methods}{%
\section{Methods}\label{methods}}

Words go here.

\hypertarget{results-and-discussion}{%
\section{Results and Discussion}\label{results-and-discussion}}

Words go here.

\hypertarget{conclusions}{%
\section{Conclusions}\label{conclusions}}

Words go here.

\hypertarget{references}{%
\section*{References}\label{references}}
\addcontentsline{toc}{section}{References}

\hypertarget{refs}{}
\begin{CSLReferences}{1}{0}
\leavevmode\hypertarget{ref-bazerghi2016role}{}%
Bazerghi, C., McKay, F.H., Dunn, M., 2016. The role of food banks in
addressing food insecurity: A systematic review. Journal of community
health 41, 732--740.

\leavevmode\hypertarget{ref-black2020examining}{}%
Black, J.L., Seto, D., 2020. Examining patterns of food bank use over
twenty-five years in vancouver, canada. VOLUNTAS: International Journal
of Voluntary and Nonprofit Organizations 31, 853--869.
doi:\url{https://doi.org/10.1007/s11266-018-0039-2}

\leavevmode\hypertarget{ref-boucher2017ontario}{}%
Boucher, B.A., Manafò, E., Boddy, M.R., Roblin, L., Truscott, R., 2017.
The ontario food and nutrition strategy: Identifying indicators of food
access and food literacy for early monitoring of the food environment.
Health promotion and chronic disease prevention in Canada : research,
policy and practice 37, 313--319.
doi:\href{https://doi.org/10.24095/hpcdp.37.9.06}{10.24095/hpcdp.37.9.06}

\leavevmode\hypertarget{ref-brunsdon2020opening}{}%
Brunsdon, C., Comber, A., 2020. Opening practice: Supporting
reproducibility and critical spatial data science. Journal of
Geographical Systems 1--20.
doi:\href{https://doi.org/10.1007/s10109-020-00334-2}{10.1007/s10109-020-00334-2}

\leavevmode\hypertarget{ref-deweese2020tale}{}%
DeWeese, J., Hawa, L., Demyk, H., Davey, Z., Belikow, A., El-geneidy,
A., 2020. A tale of 40 cities: A preliminary analysis of equity impacts
of COVID-19 service adjustments across north america. Findings.
doi:\href{https://doi.org/10.32866/001c.13395}{10.32866/001c.13395}

\leavevmode\hypertarget{ref-laborde2020poverty}{}%
Laborde, D., Martin, W., Vos, R., 2020. Poverty and food insecurity
could grow dramatically as COVID-19 spreads. International Food Policy
Research Institute (IFPRI), Washington, DC.

\leavevmode\hypertarget{ref-lovelace2021open}{}%
Lovelace, R., 2021. Open source tools for geographic analysis in
transport planning. Journal of Geographical Systems.
doi:\href{https://doi.org/10.1007/s10109-020-00342-2}{10.1007/s10109-020-00342-2}

\leavevmode\hypertarget{ref-mckay2021exploring}{}%
McKay, F.H., Bastian, A., Lindberg, R., 2021. Exploring the response of
the victorian emergency and community food sector to the COVID-19
pandemic. Journal of Hunger \& Environmental Nutrition 1--15.
doi:\href{https://doi.org/10.1080/19320248.2021.1900974}{10.1080/19320248.2021.1900974}

\leavevmode\hypertarget{ref-men2021food}{}%
Men, F., Tarasuk, V., 2021. Food insecurity amid the COVID-19 pandemic:
Food charity, government assistance and employment. Canadian Public
Policy COVID-19, e2021001.
doi:\href{https://doi.org/10.3138/cpp.2021-001}{10.3138/cpp.2021-001}

\leavevmode\hypertarget{ref-niles2020early}{}%
Niles, M.T., Bertmann, F., Belarmino, E.H., Wentworth, T., Biehl, E.,
Neff, R., 2020. The early food insecurity impacts of COVID-19. Nutrients
12, 2096.

\leavevmode\hypertarget{ref-paez2019demand}{}%
Paez, A., Higgins, C.D., Vivona, S.F., 2019. Demand and level of service
inflation in floating catchment area (FCA) methods. PloS one 14,
e0218773.
doi:\href{https://doi.org/10.1371/journal.pone.0218773}{10.1371/journal.pone.0218773}

\leavevmode\hypertarget{ref-paez2010relative}{}%
Paez, A., Mercado, R.G., Farber, S., Morency, C., Roorda, M., 2010.
Relative accessibility deprivation indicators for urban settings:
Definitions and application to food deserts in montreal. Urban Studies
47, 1415--1438.
doi:\href{https://doi.org/10.1177/0042098009353626}{10.1177/0042098009353626}

\leavevmode\hypertarget{ref-pereira2021geographic}{}%
Pereira, R.H.M., Braga, C.K.V., Servo, L.M., Serra, B., Amaral, P.,
Gouveia, N., Paez, A., 2021. Geographic access to COVID-19 healthcare in
brazil using a balanced float catchment area approach. Social Science \&
Medicine 273, 113773.
doi:\url{https://doi.org/10.1016/j.socscimed.2021.113773}

\leavevmode\hypertarget{ref-riches2002food}{}%
Riches, G., 2002. Food banks and food security: Welfare reform, human
rights and social policy. Lessons from canada? Social Policy \&
Administration 36, 648--663.

\leavevmode\hypertarget{ref-statisticscanada2020food}{}%
Statistics Canada, 2020. Food insecurity during the COVID-19 pandemic
(No. Catalogue no. 45280001).

\leavevmode\hypertarget{ref-tarasuk2014food}{}%
Tarasuk, V., Dachner, N., Loopstra, R., 2014. Food banks, welfare, and
food insecurity in canada. British Food Journal.

\leavevmode\hypertarget{ref-tarasuk2020relationship}{}%
Tarasuk, V., St-Germain, A.-A.F., Loopstra, R., 2020. The relationship
between food banks and food insecurity: Insights from canada. VOLUNTAS:
International Journal of Voluntary and Nonprofit Organizations 31,
841--852.

\leavevmode\hypertarget{ref-vanderlee2017food}{}%
Vanderlee, L., L'Abbé, M., 2017. Food for thought on food environments
in canada. Health promotion and chronic disease prevention in Canada :
research, policy and practice 37, 263--265.
doi:\href{https://doi.org/10.24095/hpcdp.37.9.01}{10.24095/hpcdp.37.9.01}

\leavevmode\hypertarget{ref-wakefield2013sweet}{}%
Wakefield, S., Fleming, J., Klassen, C., Skinner, A., 2013. Sweet
charity, revisited: Organizational responses to food insecurity in
hamilton and toronto, canada. Critical Social Policy 33, 427--450.

\end{CSLReferences}


\end{document}

